\section{Conclusions}
\label{sec:conclusions}
\ifincludetext{
In this study, 
we apply and further develop a boundary integral code for simulations of frictional fault slip, 
with poroelastic surrounding bulk, 
to study several relevant factors that may affect the stability of fault slip under fluid injection. 
First, 
we find that the fault healing, or initial slip rate under rate-and-state friction, 
affects the stability of fault slip significantly. 
A change from $V_{ini} = 10^{-22}\ \mathrm{m/s}$ to $V_{ini} = 10^{-13}\ \mathrm{m/s}$ would result in a much earlier nucleation of dynamic events and much larger spatial expansion of them, 
under the same fluid injection. 
Second, 
we further develop the code to allow for purely elastic bulk with the same fluid-transport properties, 
and confirm that poroelasticity stabilizes fault slip under fluid injection. 
We also find that with the typical length scales and properties of natural faults as well as injection time scale, 
poroelastic and elastic bulk with undrained Poisson's ratio have similar effects on dynamic fault slip. 
This is because the propagation speed of the pore pressure front is much faster than the diffusion speed of the pressure perturbations into the bulk, 
and thus the bulk s essentially undrained elastic. 
Finally, 
we study the effects of injection flux as a function of time on the stability of fault slip. 
We find that for mass-controlled injection at constant injection rate, 
higher injection rate leads to earlier and more frequent occurrences of dynamic events. However, these events have smaller spatial extent.
Motivated by that, 
we further change the injection rate from constant to intermittent in time, 
and find that with the same average injection rate and total injected fluid mass, 
intermittent injection also leads to earlier, 
more frequent but more spatially restricted dynamic events. 
This suggests that with an optimized injection rate-time profile, 
one can possibly achieve more spatially restricted and less destructive dynamic events at a given average injection rate. 
In the future, 
one can formulate an optimization over injection rate-time function, 
to achieve an objective of more stable, 
less destructive dynamic fault slip under a given average injection rate.
}
\fi